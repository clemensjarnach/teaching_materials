%\documentclass[10pt]{article}
\documentclass[twocolumn]{article} % Set document class to two columns

\usepackage{color}
\usepackage{tikz}
\usepackage[settings]{markdown}
\usepackage{titling} % Required for customizing title spacing



% Tikz settings optimized for causal graphs.
% Just copy-paste this part
\usetikzlibrary{shapes,decorations,arrows,calc,arrows.meta,fit,positioning}
\tikzset{
    -Latex,auto,node distance =1 cm and 1 cm,semithick,
    state/.style ={ellipse, draw, minimum width = 0.7 cm},
    point/.style = {circle, draw, inner sep=0.04cm,fill,node contents={}},
    bidirected/.style={Latex-Latex,dashed},
    el/.style = {inner sep=2pt, align=left, sloped}
}


\begin{document}

%######################
\title{DAGs for Causal Inference}   
\author{Clemens Jarnach}
\date{2023-04-21}										

\setlength{\droptitle}{-10em}   % This is your set screw
\maketitle
%######################


\begin{figure}[!h]
    \centering
    \caption{Mediation}
    \label{fig:Mediation}
    \begin{tikzpicture}[node distance=2.0cm and 2.0cm]
        \node (A) [label = above:{A}, point];
        \node (C) [label = above:{C}, point,  right = of A ];
        \node (B) [label = above:{B}, point, right = of C];

        \path (A) edge node[above, el] {} (C);
        \path (C) edge node[above, el] {} (B);
    \end{tikzpicture}
\end{figure}

\vspace{10mm}

\begin{figure}[!h]
    \centering
    \caption{Mutual dependence}
    \label{fig:Mutual-dependence}
    \begin{tikzpicture}[node distance=2.0cm and 2.0cm]
        \node (A) [label = below:{A}, point];
        \node (C) [label = above:{C}, point, above right = of A ];
        \node (B) [label = below:{B}, point, below right = of C];

        \path (C) edge node[above, el] {} (A);
        \path (C) edge node[above, el] {} (B);
    \end{tikzpicture}
\end{figure}

\vspace{5mm}

\begin{figure}[!h]
    \centering
    \caption{Mutual causation}
    \label{fig:Mutual-causation}
    \begin{tikzpicture}[node distance=2.0cm and 2.0cm]
        \node (A) [label = above:{A}, point];
        \node (C) [label = below:{C}, point, below right = of A ];
        \node (B) [label = above:{B}, point, above right = of C];

        \path (A) edge node[above, el] {} (C);
        \path (B) edge node[above, el] {} (C);
    \end{tikzpicture}
\end{figure}

\vspace{5mm}

\begin{figure}[!h]
    \centering
    \caption{Confounded}
    \label{fig:Confounded}
    \begin{tikzpicture}[node distance=2.0cm and 2.0cm]
        \node (A) [label = above:{A}, point];
        \node (B) [label = below:{B}, point, below right = of A];
        \node (C) [label = above:{C}, point,  right = of B ];
  
        \path (A) edge node[above, el] {} (B);
        \path (A) edge node[above, el] {} (C);
        \path (B) edge node[above, el] {} (C);
    \end{tikzpicture}
\end{figure}


\begin{figure}[!h]
    \centering
    \caption{Moderated/Interaction}
    \label{fig:Moderated}
    \begin{tikzpicture}[node distance=2.0cm and 2.0cm]
        \node (A) [label = above:{A}, point];
        \node (C) [label = above:{C}, point, above right = of A];
        \node (B) [label = above:{B}, point, below right = of C ];

        \path (A) edge node[above, el] {} (B);
    % Calculate the midpoint of the CY edge
    \coordinate (midpoint) at ($(A)!0.5!(B)$);
    % Draw the edge from M to the midpoint
    \draw (C) -- (midpoint);

    \end{tikzpicture}
\end{figure}


Figure \ref{fig:Mediation}: Mediation. This figure shows a directed acyclic graph (DAG) with three nodes A, C, and B, where A and B are not directly connected, but there is a path from A to B through C. This type of graph represents mediation, where the effect of A on B is mediated by C.

Figure \ref{fig:Mutual-dependence} : Mutual dependence. This type of graph represents mutual dependence, where A and B both depend on C.

Figure \ref{fig:Mutual-causation} : Mutual causation. This figure shows a DAG with three nodes where A is directly connected to C and B is also directly connected to C. This type of graph represents mutual causation, where A and B both cause changes in C.

Figure \ref{fig:Confounded}: Confounded. This is a DAG with three nodes A, B, and C, where A is directly connected to B, B is directly connected to C, and A is also directly connected to C. This type of graph represents confounding, where the effect of A on C is confounded by the effect of A on B and the effect of B on C.

Figure \ref{fig:Moderated}: Moderated/Interaction. This figure shows a DAG with three nodes A, B, and C, where A is directly connected to C and B is indirectly connected to C through A. This type of graph represents moderated/interaction effects, where the effect of A on C depends on the value of B.
    
%######################
\subsection*{Bibliography}
Morgan, Stephen L, and Christopher Winship. 2007. Counterfactuals and Causal Inference: Methods and Principles for Social Research. Cambridge: Cambridge University Press.


%######################
\end{document}

